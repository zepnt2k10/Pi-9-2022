 \thispagestyle{cackithitoannone}
\pagestyle{cackithitoan}
\everymath{\color{cackithi}}
\graphicspath{{../cackithi/pic/}}
%\blfootnote{{\color[named]{cackithi}$^1$Trường THPT chuyên Khoa học Tự nhiên, Đại học KHTN, Đại học Quốc Gia Hà Nội.}}
\begingroup
\AddToShipoutPicture*{\put(0,616){\includegraphics[width=19.3cm]{../bannercackithi}}} 
\AddToShipoutPicture*{\put(110,525){\includegraphics[scale=1]{../tieude.pdf}}} 
\centering
\endgroup
\vspace*{185pt}

\begin{multicols}{2}
	\textbf{\color{cackithi}Lời giải}
	\vskip 0.1cm
	\textbf{\color{cackithi}Bài $\pmb{1}$ ( Chung cho tất cả thí sinh).}
	 \vskip 0.1cm
	\textbf{\color{cackithi}Sự dán nhãn duyên dáng của một hình.}
	\vskip 0.1cm
	\textbf{\color{cackithi}Một vài ví dụ.}
	\vskip 0.1cm
	$1.$ Hình thứ nhất không phải là một dán nhãn duyên dáng bởi vì có hai trọng số giống nhau. 
	\begin{figure}[H]
		\vspace*{-10pt}
		\centering
		\captionsetup{labelformat= empty, justification=centering}
		\begin{tikzpicture}
			\draw [cackithi,line width=0.8pt] (0.,2.)-- (4.,2.);
			\draw [cackithi,line width=0.8pt] (4.,2.)-- (4.,0.);
			\draw [cackithi,line width=0.8pt] (4.,0.)-- (0.,0.);
			\draw [cackithi,line width=0.8pt] (0.,0.)-- (0.,2.);
			\draw [cackithi,line width=0.8pt] (2.,2.)-- (2.,0.);
			
			\draw [fill=cackithi] (0.,2.) circle (2.5pt);
			\draw[color=cackithi] (-0.22,2.45) node {$0$};
			\draw [fill=cackithi] (4.,2.) circle (2.5pt);
			\draw[color=cackithi] (4.24,2.45) node {$3$};
			\draw [fill=cackithi] (4.,0.) circle (2.5pt);
			\draw[color=cackithi] (4.22,-0.4) node {$5$};
			\draw [fill=cackithi] (0.,0.) circle (2.5pt);
			\draw[color=cackithi] (-0.2,-0.4) node {$7$};
			\draw [fill=cackithi] (2.,2.) circle (2.5pt);
			\draw[color=cackithi] (2.,2.45) node {$6$};
			\draw [fill=cackithi] (2.,0.) circle (2.5pt);
			\draw[color=cackithi] (1.98,-0.4) node {$1$};
			\draw (1,2.5) node[squarednode] {$6$};
			\draw (1,-0.5) node[squarednode] {$6$};
		\end{tikzpicture}
		\vspace*{-5pt}
	\end{figure}
	Hình thứ hai cũng không phải là một dán nhãn duyên dáng bởi vì có hai nhãn $7$  bằng nhau. 
	\begin{figure}[H]
		\vspace*{-10pt}
		\centering
		\captionsetup{labelformat= empty, justification=centering}
		\begin{tikzpicture}
			\draw [cackithi, line width=0.8pt] (0.,2.)-- (4.,2.);
			\draw [cackithi,line width=0.8pt] (4.,2.)-- (4.,0.);
			\draw [cackithi,line width=0.8pt] (4.,0.)-- (0.,0.);
			\draw [cackithi,line width=0.8pt] (0.,0.)-- (0.,2.);
			\draw [cackithi,line width=0.8pt] (2.,2.)-- (2.,0.);
			
			\draw [fill=cackithi] (0.,2.) circle (2.5pt);
			\draw[color=cackithi] (-0.22,2.45) node {$0$};
			\draw [fill=cackithi] (4.,2.) circle (2.5pt);
			\draw[color=cackithi] (4.24,2.45) node {$7$};
			\draw [fill=cackithi] (4.,0.) circle (2.5pt);
			\draw[color=cackithi] (4.22,-0.4) node {$4$};
			\draw [fill=cackithi] (0.,0.) circle (2.5pt);
			\draw[color=cackithi] (-0.2,-0.4) node {$7$};
			\draw [fill=cackithi] (2.,2.) circle (2.5pt);
			\draw[color=cackithi] (2.,2.45) node {$7$};
			\draw [fill=cackithi] (2.,0.) circle (2.5pt);
			\draw[color=cackithi] (1.98,-0.4) node {$2$};
		\end{tikzpicture}
		\vspace*{-10pt}
	\end{figure}
	$2.$ Hình đã cho có thể được bổ sung như sau: 
	\begin{figure}[H]
		\vspace*{-10pt}
		\centering
		\captionsetup{labelformat= empty, justification=centering}
		\begin{tikzpicture}
			\draw [cackithi,line width=0.8pt] (-1.6180339887498947,2.9021130325903073)-- (0.,2.38);
			\draw [cackithi,line width=0.8pt] (0.,2.38)-- (-1.,1.);
			\draw [cackithi,line width=0.8pt] (0.,2.38)-- (1.,1.);
			\draw [cackithi,line width=0.8pt] (0.,2.38)-- (1.618033988749895,2.9021130325903064);
			\draw [cackithi,line width=0.8pt] (0.,4.077683537175253)-- (0.,2.38);
			\draw [cackithi,line width=0.8pt] (0.,4.077683537175253)-- (-1.6180339887498947,2.9021130325903073);
			\draw [cackithi,line width=0.8pt] (-1.6180339887498947,2.9021130325903073)-- (-1.,1.);
			\draw [cackithi,line width=0.8pt] (-1.,1.)-- (1.,1.);
			\draw [cackithi,line width=0.8pt] (1.,1.)-- (1.618033988749895,2.9021130325903064);
			\draw [cackithi,line width=0.8pt] (1.618033988749895,2.9021130325903064)-- (0.,4.077683537175253);
			\draw [fill=cackithi] (-1.,1.) circle (2.5pt);
			\draw[color=cackithi] (-1.18624795654696,0.5) node[sqnode] {$3$};
			\draw (0,0.5) node[squarednode] {$2$};
			\draw (-1.5,1.7) node[squarednode] {$7$};
			\draw (-0.5,1.7) node[squarednode] {$3$};
			\draw (0.5,1.7) node[squarednode] {$1$};
			\draw (1.5,1.7) node[squarednode] {$8$};
			\draw (-1,3.5) node[squarednode] {$6$};
			\draw (-1,2.6) node[squarednode] {$10$};
			\draw (0,3.5) node[squarednode] {$4$};
			\draw (1,2.6) node[squarednode] {$9$};
			\draw (1,3.5) node[squarednode] {$5$};
			\draw [fill=cackithi] (1.,1.) circle (2.5pt);
			\draw[color=cackithi] (1.209997363286399,0.5) node[sqnode] {$1$};
			\draw [fill=cackithi] (1.618033988749895,2.9021130325903064) circle (2.5pt);
			\draw[color=cackithi] (1.8681210778885187,3.1434351104783014) node {$9$};
			\draw [fill=cackithi] (0.,4.077683537175253) circle (2.5pt);
			\draw[color=cackithi] (0.028749670410799504,4.4428075726414615) node {$4$};
			\draw [fill=cackithi] (-1.6180339887498947,2.9021130325903073) circle (2.5pt);
			\draw[color=cackithi] (-1.9624964404366394,3.0928102093550613) node {$10$};
			\draw [fill=cackithi] (0.,2.38) circle (2.5pt);
			\draw[color=cackithi] (-0.005000263671360479,2.012812318725942) node {$0$};
		\end{tikzpicture}
%		\vspace*{-5pt}
	\end{figure}
	\textbf{\color{cackithi}B. Trường hợp thẳng hàng}
	\vskip 0.1cm
	$1.$ Một dán nhãn duyên dáng của hình $L_5$	 

	\begin{figure}[H]
		\vspace*{-5pt}
		\centering
		\captionsetup{labelformat= empty, justification=centering}
		\begin{tikzpicture}
			\draw[cackithi, line width=0.8pt] (0,0) -- (5,0);
			\draw [fill=cackithi] (0,0) node[above]{$5$} circle (2.5pt);
			\draw [fill=cackithi] (1,0) node[above]{$0$} circle (2.5pt);
			\draw [fill=cackithi] (2,0) node[above]{$4$} circle (2.5pt);
			\draw [fill=cackithi] (3,0) node[above]{$1$} circle (2.5pt);
			\draw [fill=cackithi] (4,0) node[above]{$3$} circle (2.5pt);
			\draw [fill=cackithi] (5,0) node[above]{$2$} circle (2.5pt);
			
			
			\draw(0.5,-0.5) node[squarednode]{$5$};
			\draw(1.5,-0.5) node[squarednode]{$4$};
			\draw(2.5,-0.5) node[squarednode]{$3$};
			\draw(3.5,-0.5) node[squarednode]{$2$};
			\draw(4.5,-0.5) node[squarednode]{$1$};
		\end{tikzpicture}
		\vspace*{-10pt}
	\end{figure}
	Một dán nhãn duyên dáng của hình $L_6$	 

	\begin{figure}[H]
		\vspace*{-5pt}
		\centering
		\captionsetup{labelformat= empty, justification=centering}
		\begin{tikzpicture}
			\draw[cackithi, line width=0.8pt] (0,0) -- (6,0);
			\draw [fill=cackithi] (0,0) node[above]{$0$} circle (2.5pt);
			\draw [fill=cackithi] (1,0) node[above]{$6$} circle (2.5pt);
			\draw [fill=cackithi] (2,0) node[above]{$1$} circle (2.5pt);
			\draw [fill=cackithi] (3,0) node[above]{$5$} circle (2.5pt);
			\draw [fill=cackithi] (4,0) node[above]{$2$} circle (2.5pt);
			\draw [fill=cackithi] (5,0) node[above]{$4$} circle (2.5pt);
			\draw [fill=cackithi] (6,0) node[above]{$3$} circle (2.5pt);
			
			
			\draw(0.5,-0.5) node[squarednode]{$6$};
			\draw(1.5,-0.5) node[squarednode]{$5$};
			\draw(2.5,-0.5) node[squarednode]{$4$};
			\draw(3.5,-0.5) node[squarednode]{$3$};
			\draw(4.5,-0.5) node[squarednode]{$2$};
			\draw(5.5,-0.5) node[squarednode]{$1$};
		\end{tikzpicture}
		\vspace*{-10pt}
	\end{figure}
	Sự dán nhãn duyên dáng của hình $L_7$	 

	\begin{figure}[H]
		\vspace*{-5pt}
		\centering
		\captionsetup{labelformat= empty, justification=centering}
		\begin{tikzpicture}
			\draw[cackithi, line width=0.8pt] (0,0) -- (7,0);
			\draw [fill=cackithi] (0,0) node[above]{$7$} circle (2.5pt);
			\draw [fill=cackithi] (1,0) node[above]{$0$} circle (2.5pt);
			\draw [fill=cackithi] (2,0) node[above]{$6$} circle (2.5pt);
			\draw [fill=cackithi] (3,0) node[above]{$1$} circle (2.5pt);
			\draw [fill=cackithi] (4,0) node[above]{$5$} circle (2.5pt);
			\draw [fill=cackithi] (5,0) node[above]{$2$} circle (2.5pt);
			\draw [fill=cackithi] (6,0) node[above]{$4$} circle (2.5pt);
			\draw [fill=cackithi] (7,0) node[above]{$3$} circle (2.5pt);
			
			
			\draw(0.5,-0.5) node[squarednode]{$7$};
			\draw(1.5,-0.5) node[squarednode]{$6$};
			\draw(2.5,-0.5) node[squarednode]{$5$};
			\draw(3.5,-0.5) node[squarednode]{$4$};
			\draw(4.5,-0.5) node[squarednode]{$3$};
			\draw(5.5,-0.5) node[squarednode]{$2$};
			\draw(6.5,-0.5) node[squarednode]{$1$};
		\end{tikzpicture}
		\vspace*{-10pt}
	\end{figure}
	$2.$ Tương tự như các dán nhãn của hình $L_4$ và $L_6$ phía trên, ta có thể dán nhãn hình $L_{2022}$ như sau: ta đánh số các điểm từ trái qua phải dựa vào dãy sau: 
	\begin{align*}
		&0,2022,1,2021,2,2020,3,2019,4,\\
		&2018 \ldots,1000,1012,1011
	\end{align*}
	Với cách dán nhãn trên, ta nhận được các trọng số từ trái qua phải là: $2022 ,$ $2021,$ $\ldots,4,3,2,1$. Đó là một dán nhãn duyên dáng của hình $L_{2022}$.
	\vskip 0.1cm 
	\textbf{\color{cackithi}C. Trường hợp đa giác}
	\vskip 0.1cm
	$1.$ Ta có thể dán nhãn tam giác và tứ giác một cách duyên dáng như sau: 
	\begin{figure}[H]
		\centering
		\vspace*{-5pt}
		\captionsetup{labelformat= empty, justification=centering}
		\begin{tikzpicture}[scale=0.7]
			\draw [cackithi, line width=0.8pt] (1.223091301970301,2.0066510197535226)-- (0,0);
			\draw [cackithi,line width=0.8pt] (0,0)-- (4,0);
			\draw [cackithi,line width=0.8pt] (4,0)-- (1.223091301970301,2.0066510197535226);
			\draw [cackithi,line width=0.8pt] (6,0)-- (5,2);
			\draw [cackithi,line width=0.8pt] (5,2)-- (8,3);
			\draw [cackithi,line width=0.8pt] (8,3)-- (9,0);
			\draw [cackithi,line width=0.8pt] (6,0)-- (9,0);

				\draw [fill=cackithi] (1.223091301970301,2.0066510197535226) circle (2.5pt);
				\draw[color=cackithi] (1.2043242804043022,2.522744112818487) node {$0$};
				\draw [fill=cackithi] (0,0) circle (2.5pt);
				\draw[color=cackithi] (-0.259503401743597,-0.14217294955332546) node {$1$};
				\draw[color=cackithi] (0.4161093746323564,1.3779557972925676) node[squarednode] {$1$};
				\draw [fill=cackithi] (4,0) circle (2.5pt);
				\draw[color=cackithi] (4.300882838794089,-0.17970699268532278) node {$3$};
				\draw[color=cackithi] (1.8048689705162608,0.4771387621246309) node[squarednode] {$2$};
				\draw[color=cackithi] (2.8558221782121884,1.4717909051225608) node[squarednode] {$3$};
				\draw [fill=cackithi] (6,0) circle (2.5pt);
				\draw[color=cackithi] (5.708409456243992,-0.16093997111932415) node {$2$};
				\draw [fill=cackithi] (5,2) circle (2.5pt);
				\draw[color=cackithi] (4.694990291680062,2.4664430481204906) node {$4$};
				\draw[color=cackithi] (5.839778607205983,1.3591887757265688) node[squarednode] {$2$};
				\draw [fill=cackithi] (8,3) circle (2.5pt);
				\draw[color=cackithi] (8.148122259823824,3.4047941264204247) node {$0$};
				\draw[color=cackithi] (6.721828620807923,2.3538409187244986) node[squarednode] {$4$};
				\draw [fill=cackithi] (9,0) circle (2.5pt);
				\draw[color=cackithi] (9.23660951065175,-0.23600805738331884) node {$3$};
				\draw[color=cackithi] (8.78620099306778,1.8283643148765356) node[squarednode] {$3$};
				\draw[color=cackithi] (7.510043526579868,0.4771387621246309) node[squarednode] {$1$};
		\end{tikzpicture}
		\vspace*{-10pt}
	\end{figure}
	$2.$ Bằng cách thêm đỉnh số $12$ như hình dưới vào một đa giác $11$ cạnh cho trước ta nhận được một dán nhãn duyên dáng của đa giác $12$ cạnh. 
	\begin{figure}[H]
		\centering
%		\vspace*{-5pt}
		\captionsetup{labelformat= empty, justification=centering}
		\begin{tikzpicture}
			\draw [cackithi, line width=0.8pt] (-3,3)-- (-4,1);
			\draw [cackithi,dashed, line width=0.8pt] (-4,1)-- (-1,2);
			\draw [cackithi,line width=0.8pt] (-3,3)-- (-1,2);
		
				\draw [fill=cackithi] (-3,3) circle (2.5pt);
				\draw[color=cackithi] (-3.018255571945407,3.4610951911184205) node[sqnode] {$12$};
				\draw [fill=cackithi] (-4,1) circle (2.5pt);
				\draw[color=cackithi] (-4.294413038433319,0.8524791934446044) node {$0$};
				\draw[color=cackithi] (-3.6938683483213604,2.0) node[squarednode] {$12$};
				\draw [fill=cackithi] (-1,2) circle (2.5pt);
				\draw[color=cackithi] (-0.6723778761955685,1.978500487404525) node {$6$};
				\draw[color=cackithi] (-1.986069385815478,2.8) node[squarednode] {$6$};
		\end{tikzpicture}
		\vspace*{-10pt}
	\end{figure}
	$3a.$ Nếu hai đỉnh kề nhau khác tính chẵn lẻ thì hiệu của chúng là một số lẻ, do đó trọng số là số lẻ. 
	\vskip 0.1cm
	$3b.$ Hoàn toàn tương tự như trên, nếu hai đỉnh kề nhau có cùng tính chẵn lẻ thì trọng số của đoạn thẳng nối hai đỉnh đó là một số chẵn.  
	\vskip 0.1cm
	$4.$ Giả sử phản chứng rằng tồn tại dán nhãn duyên dáng đối với hình ngũ giác. Khi đó trọng số các cạnh sẽ là các số tự nhiên từ $1$ tới $5$, trong đó có $3$ số lẻ và $2$ số chẵn. Đối với $3$ cạnh có trọng số lẻ thì các đỉnh liên kết phải khác tính chẵn lẻ, nếu không trọng số sẽ là số chẵn theo chứng minh trên. Ta có hai trường hợp sau:
	\vskip 0.1cm 
	Trường hợp ${1}$: $3$ cạnh trọng số lẻ kề nhau.
	\begin{figure}[H]
		\centering
		\vspace*{-5pt}
		\captionsetup{labelformat= empty, justification=centering}
		\begin{tikzpicture}[scale=0.75]
			\draw [line width=0.8pt,color=cackithi] (-1,0)-- (1,0);
			\draw [line width=0.8pt,color=cackithi] (1,0)-- (1.618033988749895,1.9021130325903064);
			\draw [line width=0.8pt,color=cackithi] (1.618033988749895,1.9021130325903064)-- (0,3.077683537175253);
			\draw [line width=0.8pt,color=cackithi] (0,3.077683537175253)-- (-1.6180339887498947,1.9021130325903073);
			\draw [line width=0.8pt,color=cackithi] (-1.6180339887498947,1.9021130325903073)-- (-1,0);
			\draw [line width=0.8pt,color=cackithi] (4,0)-- (6,0);
			\draw [line width=0.8pt,color=cackithi] (6,0)-- (6.618033988749895,1.9021130325903064);
			\draw [line width=0.8pt,color=cackithi] (6.618033988749895,1.9021130325903064)-- (5,3.077683537175253);
			\draw [line width=0.8pt,color=cackithi] (5,3.077683537175253)-- (3.381966011250105,1.9021130325903073);
			\draw [line width=0.8pt,color=cackithi] (3.381966011250105,1.9021130325903073)-- (4,0);
		
				\draw [fill=cackithi] (-1,0) circle (2.5pt);
				\draw[color=cackithi] (-1.291689587873526,-0.6) node[squarednode] {L};
				\draw(0,0)node[below] {Chẵn};
				\draw [fill=cackithi] (1,0) circle (2.5pt);
				\draw[color=cackithi] (1.2418583235362997,-0.6) node[squarednode] {L};
				\draw(1.5,1)node[below] {Lẻ};
				\draw [fill=cackithi] (1.618033988749895,1.9021130325903064) circle (2.5pt);
				\draw[color=cackithi] (1.9,1.3) node[squarednode] {C};
				\draw(1,3)node[below] {Lẻ};
				\draw [fill=cackithi] (0,3.077683537175253) circle (2.5pt);
				\draw[color=cackithi] (0.022001921746383594,3.7) node[squarednode] {L};
				\draw(-1,3)node[below] {Lẻ};
				\draw [fill=cackithi] (-1.6180339887498947,1.9021130325903073) circle (2.5pt);
				\draw[color=cackithi] (-1.929768321117482,2.5) node[squarednode] {C};
				\draw(-1.3,1)node[below] {Chẵn};
				\draw [fill=cackithi] (4,0) circle (2.5pt);
				\draw[color=cackithi] (3.85047432121012,-0.6) node[squarednode] {C};
				\draw(5,0)node[below] {Chẵn};
				\draw(6.6,0.9)node[below] {Lẻ};
				\draw(6,3)node[below] {Lẻ};
				\draw(4,3)node[below] {Lẻ};
				\draw(3,1)node[below] {Chẵn};
				\draw [fill=cackithi] (6,0) circle (2.5pt);
				\draw[color=cackithi] (6.140050952261962,-0.6) node[squarednode] {C};
				\draw [fill=cackithi] (6.618033988749895,1.9021130325903064) circle (2.5pt);
				\draw[color=cackithi] (6.909498836467909,1.2224717677625083) node[squarednode] {L};
				\draw [fill=cackithi] (5,3.077683537175253) circle (2.5pt);
				\draw[color=cackithi] (5.032796679868039,3.7) node[squarednode] {C};
				\draw [fill=cackithi] (3.381966011250105,1.9021130325903073) circle (2.5pt);
				\draw[color=cackithi] (3.081026437004173,2.5) node[squarednode] {L};
		\end{tikzpicture}
		\vspace*{-10pt}
	\end{figure}
	Trường hợp ${2}$: $2$ cạnh trọng số lẻ kề nhau liền kề với một cạnh có trọng số chẵn. 
	\begin{figure}[H]
		\centering
		\vspace*{-5pt}
		\captionsetup{labelformat= empty, justification=centering}
		\begin{tikzpicture}[scale=0.75]
			\draw [line width=0.8pt,color=cackithi] (-1,0)-- (1,0);
			\draw [line width=0.8pt,color=cackithi] (1,0)-- (1.618033988749895,1.9021130325903064);
			\draw [line width=0.8pt,color=cackithi] (1.618033988749895,1.9021130325903064)-- (0,3.077683537175253);
			\draw [line width=0.8pt,color=cackithi] (0,3.077683537175253)-- (-1.6180339887498947,1.9021130325903073);
			\draw [line width=0.8pt,color=cackithi] (-1.6180339887498947,1.9021130325903073)-- (-1,0);
			\draw [line width=0.8pt,color=cackithi] (4,0)-- (6,0);
			\draw [line width=0.8pt,color=cackithi] (6,0)-- (6.618033988749895,1.9021130325903064);
			\draw [line width=0.8pt,color=cackithi] (6.618033988749895,1.9021130325903064)-- (5,3.077683537175253);
			\draw [line width=0.8pt,color=cackithi] (5,3.077683537175253)-- (3.381966011250105,1.9021130325903073);
			\draw [line width=0.8pt,color=cackithi] (3.381966011250105,1.9021130325903073)-- (4,0);
			
			\draw [fill=cackithi] (-1,0) circle (2.5pt);
			\draw[color=cackithi] (-1.291689587873526,-0.6) node[squarednode] {C};
			\draw(0,0)node[below] {Lẻ};
			\draw [fill=cackithi] (1,0) circle (2.5pt);
			\draw[color=cackithi] (1.2418583235362997,-0.6) node[squarednode] {L};
			\draw(1.5,1)node[below] {Chẵn};
			\draw [fill=cackithi] (1.618033988749895,1.9021130325903064) circle (2.5pt);
			\draw[color=cackithi] (1.9,1.3) node[squarednode] {L};
			\draw(1,3)node[below] {Lẻ};
			\draw [fill=cackithi] (0,3.077683537175253) circle (2.5pt);
			\draw[color=cackithi] (0.022001921746383594,3.7) node[squarednode] {C};
			\draw(-1,3)node[below] {Lẻ};
			\draw [fill=cackithi] (-1.6180339887498947,1.9021130325903073) circle (2.5pt);
			\draw[color=cackithi] (-1.929768321117482,2.5) node[squarednode] {L};
			\draw(-1.3,1)node[below] {Chẵn};
			\draw [fill=cackithi] (4,0) circle (2.5pt);
			\draw[color=cackithi] (3.85047432121012,-0.6) node[squarednode] {L};
			\draw(5,0)node[below] {Lẻ};
			\draw(6.6,0.9)node[below] {Chẵn};
			\draw(6,3)node[below] {Lẻ};
			\draw(4,3)node[below] {Lẻ};
			\draw(3,1)node[below] {Chẵn};
			\draw [fill=cackithi] (6,0) circle (2.5pt);
			\draw[color=cackithi] (6.140050952261962,-0.6) node[squarednode] {C};
			\draw [fill=cackithi] (6.618033988749895,1.9021130325903064) circle (2.5pt);
			\draw[color=cackithi] (6.909498836467909,1.2224717677625083) node[squarednode] {C};
			\draw [fill=cackithi] (5,3.077683537175253) circle (2.5pt);
			\draw[color=cackithi] (5.032796679868039,3.7) node[squarednode] {L};
			\draw [fill=cackithi] (3.381966011250105,1.9021130325903073) circle (2.5pt);
			\draw[color=cackithi] (3.081026437004173,2.5) node[squarednode] {C};
		\end{tikzpicture}
		\vspace*{-10pt}
	\end{figure}
	Khi đó sẽ tồn tại hai đỉnh được dán nhãn khác tính chẵn lẻ nhưng lại cho trọng số là số chẵn như minh họa phía trên. Điều này mâu thuẫn với tính chất đã chứng minh ở phần trước. Do vậy, không tồn tại bất cứ dán nhãn duyên dáng cho hình ngũ giác.
	\vskip 0.1cm 
	\textbf{\color{cackithi}D. Một hình đa giác với số cạnh lớn}
	\vskip 0.1cm
	$1.$ Số các đoạn thẳng bằng số cách chọn ra $2$ điểm từ $2021$ điểm, nghĩa là 
	$\dfrac{1}{2}\times 2022\times 2021=2 043 231$. Vậy, hình $K_{2022}$ được tạo thành từ $2043231$ đoạn thẳng. 
	\vskip 0.1cm
	$2a.$ Số đoạn thẳng mang trọng số lẻ chính là số những số tự nhiên lẻ của tập hợp $\{1,2,...,2043231\}$, nghĩa là bằng  $1021616$.
	\vskip 0.1cm 
	$2b.$ Vì có $p$ điểm được dán nhãn là số chẵn nên số điểm được dán nhãn là số lẻ là $2022-p$. Số những đoạn thẳng có trọng số lẻ chính là số cặp điểm được dán nhãn khác nhau về tính chẵn lẻ, do đó có tất cả  $p(2022-p)$ đoạn.
	\vskip 0.1cm
	$3.$ Giả sử hình  $K_{2022}$ có một dán nhãn duyên dáng. Khi đó có $1 021 616$ đoạn thẳng mang trọng số lẻ. Suy ra tồn tại một số tự nhiên $p$ sao cho: $p(2022-p)=1 021 616$, nghĩa là $p^2-2022p+1 021 616=0$. Phương trình này không có nghiệm nguyên, mâu thuẫn. 
	\vskip 0.1cm
	\textbf{\color{cackithi}Bài $\pmb{2}$ ( Dành cho thí sinh theo chương trình chuyên)} 
	\vskip 0.1cm
	\textbf{\color{cackithi}Phần A}
	\vskip 0.1cm
	$1a$. Các số $21$ và $136$ là phân chia được đơn nguyên vì: $21=1+2+3+4+5+6$ và $136=1+2+3+4+5+6+7+8+9+10+11+12+13+14+15+16$.
	\vskip 0.1cm 
	$1b.$ Nếu $1850$ là phân chia được đơn nguyên thì tồn tại số tự nhiên $n$ sao cho: 
	\begin{align*}
		1+2+3+\cdots+n=1850.
	\end{align*}
	Suy ra 
	\begin{align*}
		\frac{n(n+1)}{2} = 1850.
	\end{align*}
	Hay, $n^2+n-3700=0$. Phương trình bậc hai này không có nghiệm nguyên, mâu thuẫn. Do đó $1850$ không phải là số phân chia được đơn nguyên. 
	\vskip 0.1cm
	$2.$ Số tự nhiên $a$ lớn hơn hoặc bằng $3$ là một số phân chia được đơn nguyên khi và chỉ khi phương trình: $n^2+n-2a=0$ có ít nhất một nghiệm nguyên dương. Điều đó có nghĩa là biệt thức $\Delta=1+8a$ là một số chính phương và ít nhất một trong hai nghiệm \linebreak$x_1=\dfrac{-1-\sqrt{1+ 8a}}{2}$ và $x_2=\dfrac{-1+\sqrt{1+ 8a}}{2}$ là nguyên dương. Từ đó ta suy ra rằng điều kiện cần và đủ để $a$ là số phân chia được đơn nguyên là $1+8a$ là một số chính phương.
	\vskip 0.1cm  
	\textbf{\color{cackithi}Phần B}
	\vskip 0.1cm
	$1.$ Các số $9$ và $15$ là phân chia được vì $9=4+5$ và $15=7+8=4+5+6=1+2+3+4+5$. Tuy nhiên số $16$ thì không phân chia được vì:
	\begin{align*}
		&1\!+\!2\!+\!3\!+\!4\!+\!5\!<\!16\!<\!1\!+\!2\!+\!3\!+\!4\!+\!5\!+\!6;\\
		&2+3+4+5<16<2+3+4+5+6;\\
		&3+4+5<16<3+4+5+6;\\
		&4+5+6<16<4+5+6+7;\\
		&5\!+\!6\!<\!16\!<\!5\!+\!6\!+\!7; 6\!+\!7\!<\!16\!<\!6\!+\!7\!+\!8;\\
		&7+8<16<7+8+9 \text{ và } 8+9>16.
	\end{align*}
	$2.$ Gọi $n$ là số tự nhiên là lẻ và lớn hơn hoặc bằng $3$. Đặt $n=2k+1$. Khi đó $n=k+(k+1)$ do đó là phân chia được.
	\vskip 0.1cm  
	$3.$ $S=(q+1)+(q+2)+\cdots+(q+k)=(q+q+\cdots+q)+(1+2+\cdots+k)=kq+\dfrac{k(k+1)}{2}$.
	\vskip 0.1cm
	Từ đó suy ra: $2S=2kq+k(k+1)=k(2q+k+1)$. 
	\vskip 0.1cm
	$4.$ Giả sử $N=2^p$ là một lũy thừa của $2$ và là phân chia được. Theo kết câu trên, tồn tại các số tự nhiên $k$ và $q$ lớn hơn hoặc bằng $2$ sao cho: $2N=k(2q+k+1)$. Điều này là vô lý vì vế trái là một luỹ thừa của $2$ còn vế phải là tích của một số chẵn và một số lẻ lớn hơn $1$. 
	\vskip 0.1cm
	$5a.$ Ta có $56=2^3\times7$ nên $r=3$ và $m=7$. Hơn nữa $2\times 56=2^4\times7=7(2\times4+7+1)$. Do đó $56$ được viết dưới dạng tổng được định nghĩa ở ý $3)$ phần $B)$ với $k=7$ và $q=4$. Cụ thể hơn $56=5+6+7+8+9+10+11$, từ đó suy ra $56$ là số tự nhiên phân chia được. 
	\vskip 0.1cm
	$5b.$ Tương tự như trên $2\times 44=8\times11=8(2\times1+8+1)$. Do đó $44=2+3+4+5+6+7+8+9$. Ta kết luận rằng $44$ là số tự nhiên phân chia được.
	\vskip 0.1cm 
	$5c.$ Gọi $n$ là một số tự nhiên dương chẵn và không phải là lũy thừa của $2$. Đặt $n=2^r\times m$, với $m$ là một số nguyên lẻ lớn hơn hoặc bằng $3$ và $r$ một số nguyên dương. Ta suy ra $2n=2^{r+1}\times m$. Ta xét hai trường hợp sau. 
	\vskip 0.1cm
	Trường hợp $1$: Nếu $m>2^{r+1}$, tức là $m \ge 2^{r+1}+1$ và vì $m$ là một số tự nhiên lẻ nên ta suy ra tồn tại một số tự nhiên $l\ge 0$ sao cho $m=2^{r+1}+1+2l$. Khi đó $2n=2^{r+1}(2l+2^{r+1}+1)$. Do đó $n$ được viết dưới dạng tổng được định nghĩa ở ý $3)$ phần $B)$ với $k=2^{r+1}$ và $q=l$. Hay nói cách khác $n$ là số tự nhiên phân chia được. 
	\vskip 0.1cm
	Trường hợp $2$: Nếu $m<2^{r+1}$ tức là $m+1 \le 2^{r+1}$ và vì $2^{r+1}$ là một số tự nhiên chẵn nên ta suy ra tồn tại một số tự nhiên $l \ge 0$ sao cho $2^{r+1}=m+1+2l$. Khi đó $2n=m(2l+m+1)$. Do đó $n$ được viết dưới dạng tổng được định nghĩa ở ý $3)$ phần $B)$ với $k=m$ và $q=l$. Ta kết luận rằng n là số tự nhiên phân chia được.  
	\vskip 0.1cm
	Lưu ý rằng trường hợp $m=2^{r+1}$ không thể xảy ra vì $m$ là số lẻ. 
	\vskip 0.1cm
	$6.$ Từ những kết quả nhận được ở câu hỏi $2)$ và câu hỏi $5)$ ta suy ra rằng tập hợp những số tự nhiên phân chia được gồm những số tự nhiên lẻ lớn hơn hoặc bằng $3$ và những số tự nhiên chẵn không  viết được dưới dạng lũy thừa của $2$. 
	\vskip 0.1cm
	\textbf{\color{cackithi}Phần C}
	\vskip 0.1cm
	$1.$ $13$ là số tự nhiên lẻ lớn hơn $3$, nên theo kết quả trên $13$ là số tự nhiên phân chia được, hơn nữa $2\times 13=2(2\times 5+2+1)$ nên $13$ được viết dưới dạng tổng được định nghĩa ở ý $3)$ phần $B)$ với $k=2$ và $q=5$. Tức là $13=(5+1)+(5+2)$. Giả sử tồn tại một biểu diễn khác của $13$, ta suy ra tồn tại những số tự nhiên $k'\ ge 2$ và $q'$ sao cho $13=(q'+1)+(q'+2)+\cdots+(q'+k')$. Theo kết quả phần $B)$, ta có $2\times 13=k'(2q'+k'+1)$. Vì $2q'+k'+1>k'$ nên ta suy ra $k'<13$. Vì $13$ là số nguyên tố, nên ta suy ra $k'\ge 2$ là ước của $2$. Hay nói cách khác $k'=2=k$. Thay vào đẳng thức ta được $q'=5=q$. Do đó $13$ là số tự nhiên phân chia được một cách duy nhất. Tương tự $25$ là số tự nhiên phân chia được, tuy nhiên $2\times25=2\times(2\times11+2+1)=5(2\times2+5+1)$, nên theo kết quả ở phần $B)$, số tự nhiên $25$ có thể được biểu diễn dưới dạng tổng theo $2$ cách $25=12+13$ và $25=3+4+5+6+7$ nên nó không phải là số tự nhiên phân chia được một cách duy nhất.
	\vskip 0.1cm 
	$2a.$ Ta có $n=(q+1)+(q+2)+\cdots+(q+k)=k\times q+\dfrac{k(k+1)}{2}$. Nếu $k$ là số tự nhiên chẵn thì $\dfrac{k}{2}$ là một số tự nhiên, do đó $n=\dfrac{k}{2}(2q+k+1)$. Nếu $k$ là một số tự nhiên lẻ thì $\dfrac{k}{2}$ là một số tự nhiên, do đó $n=k(q+\dfrac{k+1}{2})$. Từ đó ta kết luận rằng $n$ không phải là số nguyên tố.
	\vskip 0.1cm 
	$2b$. Gọi $p$ là tố lớn hơn hoặc bằng $3$, vì $p$ là số lẻ nên theo kết quả phần $B)$ $p$ là số tự nhiên phân chia được. Hơn nữa $p=(q+1)+(q+2)$ với $q=\dfrac{p-3}{2}$. Để  ý rằng $q$ là một số tự nhiên vì $p$ là một số lẻ lớn hơn hoặc bằng $3$. Ta sẽ chứng minh biểu diễn này là duy nhất. Tương tự câu $1)$ giả sử tồn tại một biểu diễn khác của $13$, ta suy ra tồn tại những số tự nhiên $k'\ge 2$ và $q'$ sao cho $p=(q'+1)+(q'+2)+\cdots+(q'+k')$. Theo kết quả phần $B)$, ta có $2\times p=k'(2q'+k'+1)$. Vì $2q'+k'+1>k'$ nên ta suy ra $k'<p$. Vì $p$ là số nguyên tố, nên ta suy ra $k'$ là ước của $2$. Hay nói cách khác $k'=2$. Thay vào đẳng thức ta được $q'=q$. Ta kết luận rằng mọi số nguyên tố lớn hơn hoặc bằng $3$ đều phân chia được một cách duy nhất. 
	\vskip 0.1cm
	\textbf{\color{cackithi}Bài $\pmb{3}$ (Dành cho các thí sinh không theo chương trình chuyên)} 
	\vskip 0.1cm
	\textbf{\color{cackithi}Số ba}
	\vskip 0.1cm
	$1.$ Dựa vào những sơ đồ dưới đây, ta có thể khẳng định rằng cả các số tự nhiên từ $1$ đến $12$ đều có thể đạt được bằng quy tắc nêu trên. 
	\[\begin{tikzcd}[column sep = 1.35em]
		4 \arrow{r}{:\ 2}  & 2 \arrow{r}{:\ 2}& 1, &4 \arrow{r}{:\ 2}  & 2,\\[-3ex]
		4 \arrow{r}{:\ 2}  & 2 \arrow{r}{\times 3}& 6  \arrow{r}{:\  2} & 3, & 4,\\[-3ex]
		4 \arrow{r}{:\ 2}  & 2 \arrow{r}{:\ 2} &1 \arrow{r}{\times 3}& 3  \arrow{r}{+2} & 5,\\[-3ex]
		4 \arrow{r}{:\ 2}  & 2 \arrow{r}{\times 3}& 6,\\[-3ex]
		4 \arrow{r}{\times  3}  & 12 \arrow{r}{+ 2}& 14\arrow{r}{: \ 2 }& 7,\\[-3ex]
		4 \arrow{r}{:\  2}  & 2 \arrow{r}{\times 3}& 6 \arrow{r}{+2 }& 8,\\[-3ex]
		4 \arrow{r}{:\ 2}  & 2 \arrow{r}{:\ 2} &1 \arrow{r}{\times 3}& 3  \arrow{r}{\times 3} & 9, \\[-3ex]
		4 \arrow{r}{:\ 2}  & 2 \arrow{r}{\times 3} &6 \arrow{r}{\times 3}& 18  \arrow{r}{+ 2} & 20 \arrow{r}{:\ 2} & 10,\\[-3ex]
		4 \arrow{r}{:\ 2}  & 2 \arrow{r}{:\ 2} &1 \arrow{r}{\times 3}& 3  \arrow{r}{\times 3} & 9 \arrow{r}{+ 2} & 11,\\[-3ex]
		4  \arrow{r}{\times 3} &12.
	\end{tikzcd}\]
	$2.$ Dựa vào kết quả trên, ta có thể thực hiện các phép toán sao cho kết quả là $8$. Sau đó ta tiếp tục áp dụng liên tiếp các phép toán sau:
	 \[
	 \begin{tikzcd}[column sep = 1.35em]
	 	8 \arrow{r}{\times 3} & 24\arrow{r}{\times 3} & 72 \arrow{r}{+ 2} & 74 \arrow{r}{\times 3}& 222\\[-3ex]
	 	222 \arrow{r}{+ 2} & 224 \arrow{r}{\times 3} & 672 \arrow{r}{+ 2} & 674 \arrow{r}{\times 3} & 2022.
	 \end{tikzcd}
	 \]
	Ta kết luận rằng $2022$ là số tự nhiên có thể đạt được theo các quy tắc đã nêu. 
	
	$3a$. Giả sử phản chứng rằng $m$ là bội của $3$. Đặt $m=3a$. Do $m$ là số không đạt được nhỏ nhất. nhỏ nhất nên $a$ là số đạt được. Thế nhưng khi đó, ta chỉ cần áp dụng thêm phép toán Nhân $3$  với kết quả $a$ để đạt được $m$. Hay nói cách khác $m$ là số tự nhiên đạt được, mâu thuẫn. Chứng tỏ rằng giả sử phản chứng là sai, nói cách khác $m$ không phải là bội của $3$.
	\vskip 0.1cm 
	$3b$. Giả sử $m-2$ là bội của $3$, đặt $m=3b+2$. Do $m$ là số không đạt được nhỏ nhất nên $b$ là đạt được. Khi đó, chỉ cần áp dụng thêm $2$ phép toán liên tiếp Nhân $3$  rồi Cộng $2$ với từ $b$ ta thu được $m$. Hay nói cách khác $m$ là số tự nhiên đạt được, mâu thuẫn. Vậy $m-2$ không phải là bội của $3$. 
	\vskip 0.1cm
	$3c.$ Nếu $m-1$ là bội của $3$, thì tồn tại số tự nhiên dương $c$ sao cho $m=3c+1>c$. Ta suy ra $2m=3\times 2c+2$. Nếu $2c$ là số tự nhiên không đạt được bằng cách áp dụng các quy tắc như trên, thì vì $m$ là nhỏ nhất trong những số không thể đạt được nên $m\ le 2c$, hay $3c+1 \le 2c \Leftrightarrow +1 \le 0$. Điều này mâu thuẫn với điều kiện $c$ là số tự nhiên. Ngược lại, nếu $2c$ là số tự nhiên đạt được, ta áp dụng thêm $3$ phép toán liên tiếp  Nhân $3$  rồi Cộng $2$  rồi  Chia $2$ với kết quả $2c$ ta thu được $m$. Hay nói cách khác $m$ cũng là số tự nhiên đạt được. Điều này mâu thuẫn với định nghĩa của $m$. Chứng tỏ rằng $m-1$ không phải là bội của $3$.
	\vskip 0.1cm
	$3d.$ Vì trong ba số tự nhiên liên tiếp luôn có một số là bội của $3$. Nên dựa vào những kết quả trên, nếu tồn tại những số tự nhiên không đạt được bằng cách áp dụng các quy tắc đã nêu với $m$ là số tự nhiên nhỏ nhất trong số chúng, khi đó $m-2,m-1$ và $m$ đều không phải là bội của $3$. Điều này mâu thuẫn với tính chất đã nêu. Chứng tỏ, mọi số tự nhiên dương đều đạt được bằng cách áp dụng các quy tắc đã nêu. 
	\vskip 0.1cm
	\textbf{\color{cackithi}\color{cackithi}Tài liệu tham khảo}
	\vskip 0.1cm
	[$1$] Les Olympiades nationales de mathématiques | Ministère de l'Education Nationale et de la Jeunesse
	\vskip 0.1cm
	[$2$] https://www.freemaths.fr/annales-olym\\piades-mathematiques-premieres-scientifiqu\\es-s/nationales/2022
\end{multicols}